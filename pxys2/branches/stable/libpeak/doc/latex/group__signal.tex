\hypertarget{group__signal}{
\section{Signal notifications}
\label{group__signal}\index{Signal notifications@{Signal notifications}}
}


\subsection{Detailed Description}
The PEAK Library wraps signal interruptions to provide safe signal notifications (via a callback). Now you can consider handling signals like any other events (stream, timer, etc.). The API is very simple here, when you want to handle a signal (defined by a number), use \hyperlink{group__signal_ga5}{peak\_\-signal\_\-create()} and specify a pointer to a callback function. Then, schedule a task for signal notifications using \hyperlink{group__signal_ga6}{peak\_\-signal\_\-schedule()}. For convenience, especially for use with SIGPIPE, a \hyperlink{group__signal_ga8}{peak\_\-signal\_\-ignore()} method is provided to ignore a specified signal. 

\subsection*{Typedefs}
\begin{CompactItemize}
\item 
\hypertarget{group__signal_ga0}{
typedef \_\-\_\-peak\_\-signal $\ast$ \hyperlink{group__signal_ga0}{peak\_\-signal}}
\label{group__signal_ga0}

\begin{CompactList}\small\item\em Opaque type for the signal object. \item\end{CompactList}\item 
\hypertarget{group__signal_ga1}{
typedef void($\ast$ \hyperlink{group__signal_ga1}{peak\_\-signal\_\-event\_\-callback} )(\hyperlink{group__signal_ga0}{peak\_\-signal} i, int value, void $\ast$context)}
\label{group__signal_ga1}

\begin{CompactList}\small\item\em Signal notification callback type. \item\end{CompactList}\end{CompactItemize}
\subsection*{Functions}
\begin{CompactItemize}
\item 
\hyperlink{group__signal_ga0}{peak\_\-signal} \hyperlink{group__signal_ga2}{peak\_\-signal\_\-create} (int signum, \hyperlink{group__signal_ga1}{peak\_\-signal\_\-event\_\-callback} cb, void $\ast$context)
\begin{CompactList}\small\item\em Create a new signal notification object. \item\end{CompactList}\item 
int \hyperlink{group__signal_ga3}{peak\_\-signal\_\-ignore} (int signum)
\begin{CompactList}\small\item\em Ignore a signal. \item\end{CompactList}\item 
void \hyperlink{group__signal_ga4}{peak\_\-signal\_\-schedule} (\hyperlink{group__signal_ga0}{peak\_\-signal} i, \hyperlink{group__task__common_ga0}{peak\_\-task} task)
\begin{CompactList}\small\item\em Schedule a task for signal event notification. \item\end{CompactList}\item 
void \hyperlink{group__signal_ga5}{peak\_\-signal\_\-unschedule} (\hyperlink{group__signal_ga0}{peak\_\-signal} i, \hyperlink{group__task__common_ga0}{peak\_\-task} task)
\begin{CompactList}\small\item\em Unschedule a task for signal event notification. \item\end{CompactList}\end{CompactItemize}


\subsection{Function Documentation}
\hypertarget{group__signal_ga2}{
\index{signal@{signal}!peak_signal_create@{peak\_\-signal\_\-create}}
\index{peak_signal_create@{peak\_\-signal\_\-create}!signal@{signal}}
\subsubsection[peak\_\-signal\_\-create]{\setlength{\rightskip}{0pt plus 5cm}\hyperlink{group__signal_ga0}{peak\_\-signal} peak\_\-signal\_\-create (int {\em signum}, \hyperlink{group__signal_ga1}{peak\_\-signal\_\-event\_\-callback} {\em cb}, void $\ast$ {\em context})}}
\label{group__signal_ga2}


Create a new signal notification object. 

\begin{Desc}
\item[Parameters:]
\begin{description}
\item[{\em signum}]The signal number to handle. \item[{\em cb}]A pointer to your notification callback function that handles signal events that may occur once scheduled. \item[{\em context}]An extra application-defined pointer that will be passed to your signal event callback function (it's not used by the library).\end{description}
\end{Desc}
\begin{Desc}
\item[Returns:]A newly allocated peak\_\-signal reference or NULL if an error was encountered. \end{Desc}
\hypertarget{group__signal_ga3}{
\index{signal@{signal}!peak_signal_ignore@{peak\_\-signal\_\-ignore}}
\index{peak_signal_ignore@{peak\_\-signal\_\-ignore}!signal@{signal}}
\subsubsection[peak\_\-signal\_\-ignore]{\setlength{\rightskip}{0pt plus 5cm}int peak\_\-signal\_\-ignore (int {\em signum})}}
\label{group__signal_ga3}


Ignore a signal. 

Note that some signals can't be handled nor ignored, like SIGKILL.

\begin{Desc}
\item[Parameters:]
\begin{description}
\item[{\em signum}]The signal number to ignore.\end{description}
\end{Desc}
\begin{Desc}
\item[Return values:]
\begin{description}
\item[{\em 0}]The operation was successful. \item[{\em -1}]The signal can't be ignored, error might be set to indicate the error (just like sigaction(2) does). \end{description}
\end{Desc}
\hypertarget{group__signal_ga4}{
\index{signal@{signal}!peak_signal_schedule@{peak\_\-signal\_\-schedule}}
\index{peak_signal_schedule@{peak\_\-signal\_\-schedule}!signal@{signal}}
\subsubsection[peak\_\-signal\_\-schedule]{\setlength{\rightskip}{0pt plus 5cm}void peak\_\-signal\_\-schedule (\hyperlink{group__signal_ga0}{peak\_\-signal} {\em i}, \hyperlink{group__task__common_ga0}{peak\_\-task} {\em task})}}
\label{group__signal_ga4}


Schedule a task for signal event notification. 

This will enable you to receive signal notifications (using the provided callback in \hyperlink{group__signal_ga5}{peak\_\-signal\_\-create()}) within the specified task. If the task has several threads, one of its thread calls the callback.

\begin{Desc}
\item[Parameters:]
\begin{description}
\item[{\em i}]The signal object. \item[{\em task}]The task to schedule (usually \hyperlink{group__task__common_ga8}{peak\_\-task\_\-self()}). \end{description}
\end{Desc}
\hypertarget{group__signal_ga5}{
\index{signal@{signal}!peak_signal_unschedule@{peak\_\-signal\_\-unschedule}}
\index{peak_signal_unschedule@{peak\_\-signal\_\-unschedule}!signal@{signal}}
\subsubsection[peak\_\-signal\_\-unschedule]{\setlength{\rightskip}{0pt plus 5cm}void peak\_\-signal\_\-unschedule (\hyperlink{group__signal_ga0}{peak\_\-signal} {\em i}, \hyperlink{group__task__common_ga0}{peak\_\-task} {\em task})}}
\label{group__signal_ga5}


Unschedule a task for signal event notification. 

This will disable notifications for the specified signal object.

\begin{Desc}
\item[Parameters:]
\begin{description}
\item[{\em i}]The signal object. \item[{\em task}]The task to unschedule (usually \hyperlink{group__task__common_ga8}{peak\_\-task\_\-self()}). \end{description}
\end{Desc}

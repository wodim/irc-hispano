\hypertarget{group__timer}{
\section{Timer}
\label{group__timer}\index{Timer@{Timer}}
}


\subsection{Detailed Description}
Lightweight timer/repeater support within a task. Units are floats in seconds with an internal precision up to usec, but subjects to client's event handlers so don't expect a lot.\par
 For best results, please use in conjunction with the Time module ({\tt peak\_\-time$\ast$}). 

\subsection*{Typedefs}
\begin{CompactItemize}
\item 
\hypertarget{group__timer_ga0}{
typedef \_\-\_\-peak\_\-timer $\ast$ \hyperlink{group__timer_ga0}{peak\_\-timer}}
\label{group__timer_ga0}

\begin{CompactList}\small\item\em Opaque timer pointer type. \item\end{CompactList}\item 
\hypertarget{group__timer_ga1}{
typedef void($\ast$ \hyperlink{group__timer_ga1}{peak\_\-timer\_\-callback} )(\hyperlink{group__timer_ga0}{peak\_\-timer} t, void $\ast$context)}
\label{group__timer_ga1}

\begin{CompactList}\small\item\em Timer callback. \item\end{CompactList}\end{CompactItemize}
\subsection*{Functions}
\begin{CompactItemize}
\item 
\hyperlink{group__timer_ga0}{peak\_\-timer} \hyperlink{group__timer_ga2}{peak\_\-timer\_\-create} (double fire, double interval, \hyperlink{group__timer_ga1}{peak\_\-timer\_\-callback} callout, void $\ast$context)
\begin{CompactList}\small\item\em Create a timer. \item\end{CompactList}\item 
void \hyperlink{group__timer_ga3}{peak\_\-timer\_\-configure} (\hyperlink{group__timer_ga0}{peak\_\-timer} t, double fire, double interval)
\begin{CompactList}\small\item\em (Re)configure a timer. \item\end{CompactList}\item 
double \hyperlink{group__timer_ga4}{peak\_\-timer\_\-get\_\-firetime} (\hyperlink{group__timer_ga0}{peak\_\-timer} t)
\begin{CompactList}\small\item\em Get timer's fire date time. \item\end{CompactList}\item 
double \hyperlink{group__timer_ga5}{peak\_\-timer\_\-get\_\-interval} (\hyperlink{group__timer_ga0}{peak\_\-timer} t)
\begin{CompactList}\small\item\em Get timer's repeat-interval time. \item\end{CompactList}\item 
void $\ast$ \hyperlink{group__timer_ga6}{peak\_\-timer\_\-get\_\-context} (\hyperlink{group__timer_ga0}{peak\_\-timer} t)
\begin{CompactList}\small\item\em Get timer's extra application-defined context. \item\end{CompactList}\item 
void \hyperlink{group__timer_ga7}{peak\_\-timer\_\-set\_\-context} (\hyperlink{group__timer_ga0}{peak\_\-timer} t, void $\ast$context)
\begin{CompactList}\small\item\em Change the context pointer of a timer. \item\end{CompactList}\end{CompactItemize}


\subsection{Function Documentation}
\hypertarget{group__timer_ga3}{
\index{timer@{timer}!peak_timer_configure@{peak\_\-timer\_\-configure}}
\index{peak_timer_configure@{peak\_\-timer\_\-configure}!timer@{timer}}
\subsubsection[peak\_\-timer\_\-configure]{\setlength{\rightskip}{0pt plus 5cm}void peak\_\-timer\_\-configure (\hyperlink{group__timer_ga0}{peak\_\-timer} {\em t}, double {\em fire}, double {\em interval})}}
\label{group__timer_ga3}


(Re)configure a timer. 

\begin{Desc}
\item[Parameters:]
\begin{description}
\item[{\em t}]The timer reference to configure. \item[{\em fire}]Relative fire time in second. 0 means immediate fire (as soon as possible, usually during the next event loop). A value of -1.0 means never fire, and can be useful to temporarily \char`\"{}disable\char`\"{} a timer. \item[{\em interval}]Repeat interval in second. Use -1.0 for a one-shot timer. Value must be strictly positive for repeating timer. \end{description}
\end{Desc}
\hypertarget{group__timer_ga2}{
\index{timer@{timer}!peak_timer_create@{peak\_\-timer\_\-create}}
\index{peak_timer_create@{peak\_\-timer\_\-create}!timer@{timer}}
\subsubsection[peak\_\-timer\_\-create]{\setlength{\rightskip}{0pt plus 5cm}\hyperlink{group__timer_ga0}{peak\_\-timer} peak\_\-timer\_\-create (double {\em fire}, double {\em interval}, \hyperlink{group__timer_ga1}{peak\_\-timer\_\-callback} {\em callout}, void $\ast$ {\em context})}}
\label{group__timer_ga2}


Create a timer. 

Note that the timer isn't activated until you add it to the task of your choice (usually \hyperlink{group__task__common_ga8}{peak\_\-task\_\-self()}) with \hyperlink{group__task__timer_ga24}{peak\_\-task\_\-timer\_\-add()}.

\begin{Desc}
\item[Parameters:]
\begin{description}
\item[{\em fire}]Relative fire time in second. 0 means immediate fire (as soon as possible, usually during the next event loop). A value of -1.0 means never fire, and can be useful to temporarily \char`\"{}disable\char`\"{} a timer. \item[{\em interval}]Repeat interval in second. Use -1.0 for a one-shot timer. Value must be strictly positive for repeating timer. \item[{\em callout}]A pointer to your timer callback function which is triggered when the timer fires. \item[{\em context}]An extra application-defined pointer that will be passed to your timer callback function (it's not used by the library).\end{description}
\end{Desc}
\begin{Desc}
\item[Returns:]A newly allocated {\tt peak\_\-timer} reference or {\tt NULL} if the timer cannot be created. \end{Desc}
\hypertarget{group__timer_ga6}{
\index{timer@{timer}!peak_timer_get_context@{peak\_\-timer\_\-get\_\-context}}
\index{peak_timer_get_context@{peak\_\-timer\_\-get\_\-context}!timer@{timer}}
\subsubsection[peak\_\-timer\_\-get\_\-context]{\setlength{\rightskip}{0pt plus 5cm}void$\ast$ peak\_\-timer\_\-get\_\-context (\hyperlink{group__timer_ga0}{peak\_\-timer} {\em t})}}
\label{group__timer_ga6}


Get timer's extra application-defined context. 

\begin{Desc}
\item[Parameters:]
\begin{description}
\item[{\em t}]The timer reference.\end{description}
\end{Desc}
\begin{Desc}
\item[Returns:]Context pointer. \end{Desc}
\hypertarget{group__timer_ga4}{
\index{timer@{timer}!peak_timer_get_firetime@{peak\_\-timer\_\-get\_\-firetime}}
\index{peak_timer_get_firetime@{peak\_\-timer\_\-get\_\-firetime}!timer@{timer}}
\subsubsection[peak\_\-timer\_\-get\_\-firetime]{\setlength{\rightskip}{0pt plus 5cm}double peak\_\-timer\_\-get\_\-firetime (\hyperlink{group__timer_ga0}{peak\_\-timer} {\em t})}}
\label{group__timer_ga4}


Get timer's fire date time. 

\begin{Desc}
\item[Parameters:]
\begin{description}
\item[{\em t}]The timer reference.\end{description}
\end{Desc}
\begin{Desc}
\item[Returns:]Absolute time in second before the timer fires (if it is added to a task). \end{Desc}
\hypertarget{group__timer_ga5}{
\index{timer@{timer}!peak_timer_get_interval@{peak\_\-timer\_\-get\_\-interval}}
\index{peak_timer_get_interval@{peak\_\-timer\_\-get\_\-interval}!timer@{timer}}
\subsubsection[peak\_\-timer\_\-get\_\-interval]{\setlength{\rightskip}{0pt plus 5cm}double peak\_\-timer\_\-get\_\-interval (\hyperlink{group__timer_ga0}{peak\_\-timer} {\em t})}}
\label{group__timer_ga5}


Get timer's repeat-interval time. 

\begin{Desc}
\item[Parameters:]
\begin{description}
\item[{\em t}]The timer reference.\end{description}
\end{Desc}
\begin{Desc}
\item[Returns:]Interval time in second. \end{Desc}
\hypertarget{group__timer_ga7}{
\index{timer@{timer}!peak_timer_set_context@{peak\_\-timer\_\-set\_\-context}}
\index{peak_timer_set_context@{peak\_\-timer\_\-set\_\-context}!timer@{timer}}
\subsubsection[peak\_\-timer\_\-set\_\-context]{\setlength{\rightskip}{0pt plus 5cm}void peak\_\-timer\_\-set\_\-context (\hyperlink{group__timer_ga0}{peak\_\-timer} {\em t}, void $\ast$ {\em context})}}
\label{group__timer_ga7}


Change the context pointer of a timer. 

\begin{Desc}
\item[Parameters:]
\begin{description}
\item[{\em t}]The timer reference. \item[{\em context}]An extra application-defined pointer that will be passed to your timer callback function (it's not used by the library). \end{description}
\end{Desc}
